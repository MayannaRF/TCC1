\chapter{Método Proposto}\label{CAP3}
%\thispagestyle{empty}

Neste capítulo, é apresentado como serão construídos os perfis do consumo de
energia de alguns algoritmos em uma plataforma embarcada. Para executar essa tarefa com
a maior precisão, será utilizada a plataforma de hardware proposta para emulação de fontes
de energia, nomeada de Ripeto.

\section{Experimentos}
Para a realização do que foi proposto neste trabalho, deverão ser realizados
experimentos em um ambiente controlado, de forma que os resultados apresentados sejam
os mais próximos possíveis da realidade.

\subsection{Arquitetura}
Tendo em vista que o trabalho propõe fazer o perfil do consumo de energia de
alguns algoritmos em uma plataforma embarcada, uma boa alternativa de hardware para
isso seria o Ripeto, que é um plataforma baseada em FPGA Spartan6, da Xilinx, com 32 Mb
de memória DRAM acoplada. Além dele, um osciloscópio também será utilizado para
auxiliar na confirmação dos dados e na reprodução de curvas.

\subsection{Software}
Já que os sistemas embarcados tendem a estar presentes em nosso cotidiano da
sociedade, seria interessante a implementação de algoritmos que possam ser realmente
úteis em tarefas rotineiras. Sendo assim, a ideia é implementar algoritmos de
reconhecimento de padrões e algoritmos de grafos.