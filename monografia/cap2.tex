\chapter{Fundamentação teórica}
\label{CAP2}


Neste capítulo são apresentadas algumas formas de construir um perfil de consumo energético de algoritmos implementados em sistemas embarcados com exemplos de casos de usos e algumas técnicas de calcular o consumo de energia de algoritmos. Por fim, é apresentada uma técnica de emulação de captura de energia detalhando suas especificidades.


\section{Sistemas embarcados}
A enorme de evolução dos sistemas embarcados fez com que a sociedade se tornar, de certa forma, dependente deles, já que eles são essenciais para diversos setores, desde o automotivo até a medicina chegando até mesmo no vestuário. Esse progresso  se deve também ao crescimento da Internet das Coisas (IoT) 	que ainda deve crescer muito nos próximos anos como mostra graficamente a Figura 1 da IHS (2016).  
\begin{figure}[!ht]
\centering
\includegraphics[scale=.65]{figures/grafico.jpg}
\caption{ Figura 1 - The IoT market will be massive. Fonte IHS (2016) } \label{Fig:1}
\end{figure}

Os avanços da microeletrônica proporcionaram o surgimento de diversos microprocessadores, microcontroladores, processador de sinais digitais (DSP) e FPGA baseado em VLSI. Essa plataformas tem um papel primordial para o  IoT, já que é com elas  que acontece a automação de diversos sistemas. Cada um desses  microeletrônicos tem suas peculiaridades porém todos eles têm algo,muito nítido, em comum eles depende de energia para seu funcionando. Saber quanto é seu consumo energético se torna importante para seu total aproveitamento, uma parte desse consumo vem do algoritmo que está implementado no sistema. Para saber o consumo energético de determinados algoritmos existem alguns métodos que serão abordados a seguir.

\subsection{Modelos Matematicos}
Construir um modelo matemático que possa determinar o desempenho energético um sistema sem que ele esteja montado fisicamente seria muito interessante, já que assim simplificaria e muito o processo de produção. Um modelo com essas características já foi construído para redes de sensores sem fio. O modelo proposto aborda uma parte importante do processo operacional das redes de sensores visuais sem fio, que é o processamento interno nos nós sensores (CERQUEIRA; COSTA, 2019).

\subsection{Simuladores}
Existem alguns simuladores de consumo energético um desse é o Sim-Panalyzer.  O SimpleScalar é um simulador de arquitetura computacional que modela um computador virtual com CPU, cache e hierarquia de memória e com base nesse modelo consegue simular programas reais executando sobre a plataforma especificada (LIMA; et al., 2012).

\subsection{Osciloscopio }
Dentre de todos os métodos relatados acima o osciloscópio é o mais tradicional quando se trata de medição de consumo de energia. Para a medição do consumo de energia foi usado o osciloscópio para medir a corrente consumida pelo processador ao executar o algoritmo desejado. Para medir a corrente, foi inserido um resistor de 0.333 Ohm conectado em série com o cabo de alimentação ATX12V da placa mãe e medida a diferença de voltagem da entrada e saída do resistor de shunt que foi inserido em série no circuito da placa principal,conforme se observa na Figura 2 ( NETO; MORENO; MATOS, 2011) .

\begin{figure}[!ht]
\centering
\includegraphics[scale=.65]{figures/ociloscopio.png}
\caption{ Figura 2 - Esquema utilizado para as medições com o Osciloscópio. Retirado de  ( Neto; Moreno; Matos, 2011). } \label{Fig:1}
\end{figure}


\subsection{Ripeto}
 Ripeto, uma plataforma de avaliação para EHES que pode imitar o comportamento do transdutor de captação de energia e registrar em rastreamentos de energia de memória e dados do analisador lógico para assegurar adequadamente o desempenho de um EHES (ALCANTARA; DE LIMA; FURTADO, 2019). O Ripeto é baseado em um FPGA Sparten6, da Xilinx, com 32 Mb de memória de DRAM acoplada.

\begin{figure}[!ht]
\centering
\includegraphics[scale=.65]{figures/ripeto.png}
\caption{Figura 3 - Arquitetura da Plataforma. Retirado de (ALCANTARA; DE LIMA; FURTADO, 2019) } \label{Fig:1}
\end{figure}

O elemento central, o FPGA é onde os  componentes lógicos são implementados. A intenção com essa plataforma é capturar os  perfis de fornecimento de energia de fontes renováveis e reproduzi los de forma consistente.