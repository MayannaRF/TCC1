\chapter*{Resumo}

\noindent A tecnologia está onipresente atualmente, isso se deve em muito por causa da internet
das coisa, já que  ela possibilitou a automatização de diversos equipamentos. Apesar dessa automação os equipamentos não ficaram totalmente independentes uma vez que ainda existe a dependência da eletricidade. Alguns desses equipamentos tem como fonte de energia bateria, que tem sua vida útil limitada, sendo assim conhecer o quanto será consumido de energia se torna imprescindível. Uma boa parte do consumo energético vem dos algoritmos que estão funcionando na plataforma. Atualmente já existem diversos modelos para se calcular o consumo energético de algoritmos em plataformas embarcadas, porém não são completamente eficientes, sendo assim a proposta seria juntar mais de um desses modelos para uma maior eficiência e precisão nas medições. Para realizar a experimentações o presente trabalho propõe explora uma plataforma chamada Repito
para que assim possa traçar o perfil do consumo de energia de alguns algoritmos
implementados em sistemas embarcados.



\textbf{Palavras-chaves}: Sistemas embarcados, perfil energético, internet das coisas.
