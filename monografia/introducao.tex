-\chapter{Introdução}\label{CAP:introducao}
%\thispagestyle{empty}

A tecnologia está cada vez mais presente na vida da sociedade atual. As evoluções dessas tecnologias nas últimas décadas foram enormes. Hoje, elas estão presentes em diversos âmbitos da sociedade, como na agricultura, no comércio, nos equipamentos médicos, no monitoramento de rodovias, nas casas e até no corpo humano. Nesse cenário, os sistemas embarcados tornam-se cada vez mais presentes no cotidiano.

Os sistemas embarcados têm muitas características que os beneficiam, tais quais: o preço acessível, a portabilidade e a capacidade de processamento, que só vem aumentando com o passar dos anos, além do baixo consumo de energia. 
            
Muitos desses equipamentos irão passar longos períodos sem nem uma intervenção humana. Sendo assim, apesar de seu pequeno consumo de energia, é muito importante ter uma noção exata de quanto ele será, para que, dessa forma, seja possível aumentar a autonomia do sistema e, consequentemente, sua vida útil. O consumo de energia desses dispositivos pode variar muito em função das operações atreladas a aplicação, dos modos de operação e dos periféricos utilizados. O que também pode contribuir para o gasto energético são os softwares utilizados no sistema.	
      
\section{Motivaçâo}
Diante da proeminente proliferação dos sistemas embarcados na sociedade e de todos os desafios existentes para torná-los mais eficientes quanto ao consumo energético, torna-se necessário ter conhecimento do perfil energético de determinados algoritmos que nos dispositivos serão utilizados. Com a construção dos perfis energéticos, seria mais fácil comparar e decidir qual algoritmo utilizar em determinados dispositivos e situações e, por consequência, o sistema desenvolvido seria mais eficiente.

\section{Objetivo Geral}
Traçar, de forma consistente, um perfil de consumo de energia de alguns algoritmos em um sistema embarcado em diferentes situações, permitindo, assim, uma avaliação apurada para uso em plataformas de simulação.

\subsection{Objetivos Especificos}
Os objetivos específicos deste trabalho são os seguintes:
\begin{itemize}
    \item 	Utilizar a plataforma apresentada no artigo {\itshape An FPGA-based evaluation platform for energy harvesting embedded systems} (ALCÂNTARA; DE LIMA; FURTADO, 2019) para fazer os experimentos;
    
    \item 	Empregar os perfis de consumo de energia gerados em diferentes condições, para uso como fonte de dados para simulações;
     
     \item 	Comparar os perfis de consumo de um mesmo algoritmo em diferentes condições.

\end{itemize}


 
\section{Contribuicoes}




\section{Producao cientifica}


\section{Organizacao da tese}

\noindent \textbf{Capitulo \ref{CAP2}}: descricao...

\noindent \textbf{Capitulo \ref{CAP3}}: descricaoo...

\noindent \textbf{Capitulo \ref{CAP4}}: descricao...

\noindent \textbf{Capitulo \ref{CAP5}}: descricao...